\section{Definition of objectives}


The project objectives are divided into two separate categories called basic and advanced goals. The separate goals are planned to make sure at least the minimal wanted result for the project is achieved. If there is still time left to work on the project after the basic requirements are met, then the tasks defined in the advanced goal will be worked on.

\section{The basic goal}

The basic goal is the minimal wanted result for the project. The basic goal requirements are planned so that each project member gains 5 credits for the project work. One credit amounts to $26+\frac{2}{3}$ working hours. This means that the work spent on the basic goal will be approximately 133.3 hours for each project member. The general idea of the basic goal is to achieve a broad understanding of the possibilities with the basic idea that is being researched and see if a plausable method for the defined task could be further developed. The possible application and end result of the basic goal does not need to be the best possible tool for the given task, but to give insight whether this kind of method is worth investigating further. 

The objectives of the basic goal are as follows:

\begin{itemize}
	\item Project initialization and planning
	\item Preliminary study of open-source video object detection and object tracking methods that could be applied in the project
	\item Designing the scenes that are to be recorded using a 360-video camera and the integrated microphone array
	\item Recording the designed scenes
	\item Developing a set of Python scripts for handling the video data frames (basic data handling methods for video)
	\item Developing a set of Python scripts for applying object detection (and possibly object tracking) for the video data frames
	\item Developing a method for outputting the detected visual objects, their locations and the time interval of detection to be used together with spatial audio detection
	\item Testing the developed video handling scripts with the recorded video data
	\item Developing a method for outputting the detected sound events, their locations and the time interval of detection
	\item Manually mapping visual object labels with the sound event labels to be used for automated annotation
	\item Documenting
	\item Testing the method against manual annotation
\end{itemize} 

\section{The advanced goal}

The advanced goal is meant to be worked on in case the basic goal is achieved and there is still time for further development. The advanced goal requirements are planned so that each project member gains 6 credits for the project work. This will amount to a total of 160 hours of work for each project member. 

The objectives of the advanced goal are as follows

\begin{itemize}
	\item Everything listed under the basic goal requirements
	\item More complicated scenes with more events and more sounds
	\item Fusion of audio and video
\end{itemize} 


\section{Deliverables}

The deliverables for the project will be this project plan document, the final report of the project work, the presentations given during the course, and the source codes of the developed application.