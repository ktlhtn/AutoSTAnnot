%%%%% PREAMBLE %%%%%

%%%%% Document class declaration.
% The possible optional arguments are
%   finnish - thesis in Finnish (default)
%   english - thesis in English
%   numeric - citations in numeric style (default)
%   authoryear - citations in author-year style
%   apa - citations in APA 7 (available only in English)
%   ieee - citations in IEEE style
%       apa and ieee provide biblatex basic styles as is!
%   draft - for faster non-final works, also skips images
%           (recommended, remove in final version)
%   programs - if you wish to display code snippets
% Example: \documentclass[english, authoryear]{tauthesis}
%          thesis in English with author-year citations
\documentclass[english]{tauthesis}

% The glossaries package throws a warning:
% No language module detected for 'finnish'.
% You can safely ignore this. All other
% warnings should be taken care of!

%%%%% Your packages.
% Before adding packages, see if they can be found
% in tauthesis.cls already. If you're not sure that
% you need a certain package, don't include it in
% the document! This can dramatically reduce
% compilation time.

% Graphs
% \usepackage{pgfplots}
% \pgfplotsset{compat=1.15}

% Subfigures and wrapping text
% \usepackage{subcaption}

% Mathematics packages
\usepackage{amsmath, amssymb, amsthm}
%\usepackage{bm}

% Chemistry packages
% Newest mhchem is attached for compatibility
% \usepackage{chemfig}
% \usepackage[version=4]{mhchem}

% Text hyperlinking
% \usepackage{hyperref}
% \hypersetup{hidelinks}

% (SI) unit handling
% \usepackage{siunitx}

%\sisetup{
%    detect-all,
%    math-sf=\mathrm,
%    exponent-product=\cdot,
%    output-decimal-marker={,} % for theses in FINNISH!
%}

%%%%% Your commands.
\usepackage{eurosym}
% Print verbatim LaTeX commands
\newcommand{\verbcommand}[1]{\texttt{\textbackslash #1}}

% Basic theorems in Finnish and in English.
% Remove [chapter] if you wish a simply
% running enumeration.
% \newtheorem{lause}{Lause}[chapter]
% \newtheorem{theorem}[lause]{Theorem}

% \newtheorem{apulause}[lause]{Apulause}
% \newtheorem{lemma}[lause]{Lemma}

% Use these versions for individually
% enumerated lemmas
% \newtheorem{apulause}{Apulause}[chapter]
% \newtheorem{lemma}{Lemma}[chapter]

% Definition style
% \theoremstyle{definition}
% \newtheorem{maaritelma}{Määritelmä}[chapter]
% \newtheorem{definition}[maaritelma]{Definition}
% examples in this style

\newcommand{\highlight}{\textcolor{red}} 

%%%%% Citation information.
\addbibresource{tex/references.bib}

\begin{document}

%%%%% FRONT MATTER %%%%%

\frontmatter

%%%%% Thesis information and title page.

% The titles of the work. If there is no subtitle,
% leave the arguments empty. Pass the title in
% the primary language as the first argument
% and its translation to the secondary language
% as the second.
\title{Signal Processing Innovation Project}{Signal Processing Innovation Project}
\subtitle{Automated spatiotemporal annotation of sound objects in a scene}{Automated spatiotemporal annotation of sound objects in a scene}

% The author name.
\author{Einari Vaaras, Mehmet Aydin and Kalle Lahtinen}

% The examiner information.
% If your work has multiple examiners, replace with
% \examiner[<label>]{<name> \\ <name>}
% where <label> is an appropriate (plural) label,
% e.g. Examiners or Tarkastajat, and <name>s are
% replaced by the examiner names, each on their
% separate line.
\examiner[Clients]{Archontis Politis and Tuomas Virtanen}

% The finishing date of the thesis (YYYY-MM-DD).
\finishdate{2021}{05}{03}

% The type of the thesis (e.g. Kandidaatintyö
% or Master of Science Thesis) in the primary
% and the secondary languages of the thesis.
\thesistype{Final Report}{Final Report}

% The faculty and degree programme names in
% the primary and the secondary languages of
% the thesis.
\facultyname{Faculty of Information Technology and Communication Sciences}{Faculty of Information Technology and Communication Sciences}
\programmename{}{}

% The keywords to the thesis in the primary and
% the secondary languages of the thesis
\keywords%
    {}
    {}

\maketitle

\clearpage
\markright{Version History}
\section*{Version History}
\label{ch:version_history}
\begin{center}
	\begin{tabular}{ c c c }
		28.4.2021 & Kalle L & Initial \\ 
		2.5.2021 & Einari V & Added initial versions of 2.2, 2.3, 3.1-3.8, 4 and 5 \\
		2.5.2021 & Mehmet A & Added the section about Deep SORT to 3.1 \\
		3.5.2021 & Einari V & Added more content to 3.7 and 4\\
		6.5.2021 & Kalle L & Added abstract, more to introduction, 2, 3.1, 3.7\\
	\end{tabular}
\end{center}

%%%%% Abstracts and preface.

% Write the abstract(s) and the preface
% into a separate file for the sake of clarity.
% Pass the appropriate file name as the first
% argument to these commands. Put the \abstract
% in the primary language first and the
% \otherabstract in the secondary language second.
% Those who do not speak Finnish only need the
% first abstract. The second argument of
% the \preface command takes the place where
% the thesis was signed in.
\abstract{tex/abstract.tex}
%\otherabstract{tex/abstract.tex}
%\preface{tex/alkusanat.tex}{Tampereella}

%%%%% Table of contents.

\tableofcontents

%%%%% Lists of figures, tables, listings and terms.

% Print the lists of figures and/or tables.
% Uncomment either of these commands as required.
% Both are optional, but if there are many important
% figures/tables, listing them may be a good idea.

% \listoffigures
% \listoftables
% \lstlistoflistings

%%%%% MAIN MATTER %%%%%

\mainmatter

% Write each of the chapters of the thesis
% into a separate file for the sake of clarity.
% They can be \input as shown below. Give both
% the chapters and their files as descriptive
% names as possible.
\chapter{Introduction}
\label{ch:project_background}
This project was the course work for a Tampere University course 'Signal Processing Innovation Project' designed for students majoring in signal processing and machine learning. The course was completed during the spring of the year 2021. The students working on the project were Einari Vaaras, Mehmet Aydin and Kalle Lahtinen. The goal of the project was to study the possibilities for automatically annotating spatial and temporal dimensions of sound objects detected in an audiostream with the help of a 360-degree video recording. The premise was that the objects are over two meters away from the the camera, so the project involves limited elevation angles. Another basic element was that including multiple sound sources is relevant, but there is no need to go to extremes, such as two similar objects being very close to each other.

The need for such a tool rises from the research of spatial audio in which data intensive modelling methods are heavily used. The goal of the project was to provide a proof-of-concept application which would improve the quality of spatial audio data used in the training of such models and reduce manual labour related to the annotation of the data. The project was implemented using open-source tools. The main tool for development was Python and available signal processing and machine learning libraries. In addition to Python, Matlab-scripts provided by the client were also used for the audio processing. The resulting application and its source code are shared as open-source (\highlight{source to our own repo here}) for further research use under the Aladdin Free Public License.

The project proceeded according to the project plan made at the very beginning of the course. The end result of the project work was two different proof-of-concept processing pipelines that combine detections made from 360-video recordings with spatial audio recordings from the same scene. The pipeline outputs provide annotated spatio-temporal audio activity data that could be used for model training. The training data created with the pipelines were not yet tested in actual model training, but the findings from this project indicate that the basic method could be useful and the topic should be further researched.  

\chapter{Summary of the Project}
\label{ch:project_organisation}
\section{Project background}

\highlight{(Something here)}

\section{Project objectives, deliverables and planned timetable}

The overall goal of the project was to develop a proof-of-concept for providing spatiotemporal labels for audio events using a 360-degree camera and a microphone array. In addition to the course-specific deliverables such as the project plan report, the final report and the final presentation, the main deliverable of the project was agreed with the clients to be a GitHub repository containing the source codes that were created for the components of the project. Additionally, a thorough documentation of the source codes was agreed to be delivered on the GitHub repository. Practically the only timetable for the project was that some kind of functioning pipeline would be finished by the end of the course. On a weekly basis, the tasks for the next week were agreed on together with the clients on a Teams meeting.

\section{Project organization}

The project members were Einari Vaaras, Mehmet Aydin and Kalle Lahtinen. The open tasks related to the project were discussed on a daily basis in a project-specific Telegram channel. Before the project, there were no predefined roles for the project members, and the aim was to divide weekly tasks based on open discussion and equal contribution. During the project, each group member specialized in different categories of the project. Einari handled parts related to audio powermaps, beamforming, bounding box cleaning and mapping class labels. Mehmet specialized in experimenting with the recording devices such as providing test-scenes-to-be-experimented with on a weekly basis. Additionally, he studied about tracking methods and their implementations, especially Deep Sort \highlight{(reference here!)}. Kalle took care of the video object detection pipeline, as well as defining a standardized CSV file output and producing audio event detections based on audio powermaps. In problematic situations such as code debugging, all group members participated in helping the group. In addition, course-related activities such as writing reports and planning presentations were carried out by all group members. Furthermore, all group members participated in the final recordings for the project, which are further discussed at the end of Section \ref{sec_implementation_steps}.




\chapter{Realized Project Implementation}
\label{ch:project_objectives}
\section{Implementation Steps} \label{sec_implementation_steps}

\section{Meetings, Week Reports and Inspections}

During the project, almost every Friday at 14:00 a MS Teams meeting was organized together with the clients to discuss updates about the project from the past week, and to set goals for the following week.

\section{Deliverables}

The main deliverables of the present project were the source codes and their thorough documentation created during the project which are provided in a GitHub repository. Additionally, the present final report is the main deliverable course-wise.

\section{Timetable and Workload}

\section{Budget}

\section{Problems, Delays and Changes in Project Organization and Plans}

\section{Lessons learned}

\section{Future Development Needs}

Implementation should be tested more vigorously: more recordings, testing the produced data in actual training

trying out the system with a microphone array with a better spatial resolution: audio powermap can be improved

making the pipeline end-to-end

adding more intelligent object tracking methods

improving the video detector: more objects, better recognition, more carefully designing scenes so that objects are better detected (not very close to the camera)




\chapter{Project Results and Conclusions}
\label{ch:project_results}
The main goal of the project was to get a working pipeline for automatically generating spatiotemporal labels for audio events, and the project group managed to get two functioning pipelines for the task successfully. As an outcome, a GitHub repository containing all source codes that were created during the project was created together with a thorough documentation of how the provided codes can be used for the task. Overall, the project provided valuable insight into the topics of object detection for 360-degree videos and spatial sound recognition for both the group members and the clients. However, as listed in Section \ref{sec_future_development_needs}, there are multiple areas of the project that can be vastly improved. Perhaps one of the main outcomes of the project was to gain knowledge for the clients of what is currently working and what is not, and what are the limitations of the current pipelines and the equipment that was used.

\highlight{(Additions?)}

\chapter{Comments and Opinions on the Course}
\label{ch:implementation}



% Add chapters similarly.


%%%%% Bibliography/references.

% Print the bibliography according to the
% information in ./tex/references.bib and
% the in-line citations used in the body of
% the thesis.
% \emergencystretch=2em
\printbibliography[heading=bibintoc]

\end{document}
