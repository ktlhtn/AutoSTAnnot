%%%%% PREAMBLE %%%%%

%%%%% Document class declaration.
% The possible optional arguments are
%   finnish - thesis in Finnish (default)
%   english - thesis in English
%   numeric - citations in numeric style (default)
%   authoryear - citations in author-year style
%   apa - citations in APA 7 (available only in English)
%   ieee - citations in IEEE style
%       apa and ieee provide biblatex basic styles as is!
%   draft - for faster non-final works, also skips images
%           (recommended, remove in final version)
%   programs - if you wish to display code snippets
% Example: \documentclass[english, authoryear]{tauthesis}
%          thesis in English with author-year citations
\documentclass[english]{tauthesis}

% The glossaries package throws a warning:
% No language module detected for 'finnish'.
% You can safely ignore this. All other
% warnings should be taken care of!

%%%%% Your packages.
% Before adding packages, see if they can be found
% in tauthesis.cls already. If you're not sure that
% you need a certain package, don't include it in
% the document! This can dramatically reduce
% compilation time.

% Graphs
% \usepackage{pgfplots}
% \pgfplotsset{compat=1.15}

% Subfigures and wrapping text
% \usepackage{subcaption}

% Mathematics packages
\usepackage{amsmath, amssymb, amsthm}
%\usepackage{bm}

% Chemistry packages
% Newest mhchem is attached for compatibility
% \usepackage{chemfig}
% \usepackage[version=4]{mhchem}

% Text hyperlinking
% \usepackage{hyperref}
% \hypersetup{hidelinks}

% (SI) unit handling
% \usepackage{siunitx}

%\sisetup{
%    detect-all,
%    math-sf=\mathrm,
%    exponent-product=\cdot,
%    output-decimal-marker={,} % for theses in FINNISH!
%}

%%%%% Your commands.
\usepackage{eurosym}
% Print verbatim LaTeX commands
\newcommand{\verbcommand}[1]{\texttt{\textbackslash #1}}

% Basic theorems in Finnish and in English.
% Remove [chapter] if you wish a simply
% running enumeration.
% \newtheorem{lause}{Lause}[chapter]
% \newtheorem{theorem}[lause]{Theorem}

% \newtheorem{apulause}[lause]{Apulause}
% \newtheorem{lemma}[lause]{Lemma}

% Use these versions for individually
% enumerated lemmas
% \newtheorem{apulause}{Apulause}[chapter]
% \newtheorem{lemma}{Lemma}[chapter]

% Definition style
% \theoremstyle{definition}
% \newtheorem{maaritelma}{Määritelmä}[chapter]
% \newtheorem{definition}[maaritelma]{Definition}
% examples in this style

%%%%% Citation information.
\addbibresource{tex/references.bib}

\begin{document}

%%%%% FRONT MATTER %%%%%

\frontmatter

%%%%% Thesis information and title page.

% The titles of the work. If there is no subtitle,
% leave the arguments empty. Pass the title in
% the primary language as the first argument
% and its translation to the secondary language
% as the second.
\title{Signal Processing Innovation Project}{Signal Processing Innovation Project}
\subtitle{Automated spatiotemporal annotation of sound objects in a scene}{Automated spatiotemporal annotation of sound objects in a scene}

% The author name.
\author{Einari Vaaras, Mehmet Aydin and Kalle Lahtinen}

% The examiner information.
% If your work has multiple examiners, replace with
% \examiner[<label>]{<name> \\ <name>}
% where <label> is an appropriate (plural) label,
% e.g. Examiners or Tarkastajat, and <name>s are
% replaced by the examiner names, each on their
% separate line.
\examiner[Clients]{Archontis Politis and Tuomas Virtanen}

% The finishing date of the thesis (YYYY-MM-DD).
\finishdate{2020}{02}{11}

% The type of the thesis (e.g. Kandidaatintyö
% or Master of Science Thesis) in the primary
% and the secondary languages of the thesis.
\thesistype{Project plan}{Project plan}

% The faculty and degree programme names in
% the primary and the secondary languages of
% the thesis.
\facultyname{Faculty of Information Technology and Communication Sciences}{Faculty of Information Technology and Communication Sciences}
\programmename{}{}

% The keywords to the thesis in the primary and
% the secondary languages of the thesis
\keywords%
    {}
    {}

\maketitle

\clearpage
\markright{Version History}
\section*{Version History}
\label{ch:version_history}
\begin{center}
	\begin{tabular}{ c c c }
		28.4.2021 & Kalle L & Initial \\ 
		2.5.2021 & Einari V & Added initial versions of 2.2, 2.3, 3.1-3.8, 4 and 5 \\
		2.5.2021 & Mehmet A & Added the section about Deep SORT to 3.1 \\
		3.5.2021 & Einari V & Added more content to 3.7 and 4\\
		6.5.2021 & Kalle L & Added abstract, more to introduction, 2, 3.1, 3.7\\
		6.5.2021 & Einari V & Finalized the report\\
	\end{tabular}
\end{center}

%%%%% Abstracts and preface.

% Write the abstract(s) and the preface
% into a separate file for the sake of clarity.
% Pass the appropriate file name as the first
% argument to these commands. Put the \abstract
% in the primary language first and the
% \otherabstract in the secondary language second.
% Those who do not speak Finnish only need the
% first abstract. The second argument of
% the \preface command takes the place where
% the thesis was signed in.
\abstract{tex/abstract.tex}
%\otherabstract{tex/abstract.tex}
%\preface{tex/alkusanat.tex}{Tampereella}

%%%%% Table of contents.

\tableofcontents

%%%%% Lists of figures, tables, listings and terms.

% Print the lists of figures and/or tables.
% Uncomment either of these commands as required.
% Both are optional, but if there are many important
% figures/tables, listing them may be a good idea.

% \listoffigures
% \listoftables
% \lstlistoflistings

%%%%% MAIN MATTER %%%%%

\mainmatter

% Write each of the chapters of the thesis
% into a separate file for the sake of clarity.
% They can be \input as shown below. Give both
% the chapters and their files as descriptive
% names as possible.
\chapter{Project Background}
\label{ch:project_background}
This project is the course work for a Tampere University course 'Signal Processing Innovation Project' designed for students majoring in signal processing and machine learning. The course will be completed during spring of the year 2021. The students working on the project are Einari Vaaras, Mehmet Aydin and Kalle Lahtinen. The goal of the project is to study the possibilities for automatically annotating spatial and temporal dimensions of sound objects detected in an audiostream with the help of a 360-degree video recording. The premise is that the objects are over two meters away from the the camera, so the project involves limited elevation angles. Another basic element is that including multiple sound sources is relevant, but there is no need to go to extremes, such as two similar objects being very close to each other.

The need for such a tool rises from the research of spatial audio in which data intensive modelling methods are heavily used. The goal of the project is to provide a proof-of-concept application which would improve the quality of spatial audio data used in the training of such models and reduce manual labour related to the annotation of the data. The project will be implemented using open-source tools. The main tool for development will be Python and available signal processing and machine learning libraries. The resulting application and its source code will be shared as open-source for further research use.

\chapter{Project Organisation}
\label{ch:project_organisation}
\section{Project members}

The project members are Einari Vaaras, Mehmet Aydin and Kalle Lahtinen. There are no defined roles for the project members. The open tasks related to the project are discussed daily in a project-specific Telegram-channel. The workload is divided within the group on a weekly basis using a democratic decision making process in which open discussion and equality are emphasized. The goal of the project is to achieve concrete results in a way that every group member gets to participate and learn something new. 

\section{Client}

The project is an assignment given by Archontis Politis (archontis.politis@tuni.fi) and Tuomas Virtanen (tuomas.virtanen@tuni.fi), both working at the Audio Research Group (ARG) at Tampere University.



\chapter{Project Objectives}
\label{ch:project_objectives}
\section{Definition of objectives}

The objectives of different interest groups (project members, client and other
groups)
List here all possible goals you have set up with your client. Justify which
achievements are needed to receive the credits (and possibly, which extra
achievements give some extra credits).
Technical objectives (if applicable)
The required and unwanted properties of the objectives


The project objectives are divided into separate categories called basic and advanced goals. The separate goals are planned to make sure atleast the minimal wanted result for the project is achieved. If there is still time left to work on the project after the basic requirements are met the tasks defined in the advanced goal will be worked on. 

\section{The basic goal}

The basic goal is the minimal wanted result for the project. The basic goal requirements are planned so that each project member gains 5 credits for the project work. One credit amounts to 26+2/3 working hours. This means that the work spent on the basic goal will be approximately 133.3 hours for each project member. 

The objectives of the basic goal are as follows

\begin{itemize}
	\item Project initialization and planning
	\item Preliminary study of open source video detection methods that could be applied in the project
	\item Designing the scenes that are to be recorded using a 360 video camera and the integrated microphone array
	\item Recording the designed scenes
	\item Developing a set of python scripts for handling the video data frames (basic data handling methods for video)
	\item Developing a set of python scripts for applying object detection for the video data frames
	\item Developing a method for outputting the detected visual objects, their locations and the time interval of detection to be used together with spatial audio detection
	\item Testing the developed video handling scripts with the recorded video data
	\item Developing a method for outputting the detected sound events, their locations and the time interval of detection to be used together with spatial audio detection
	\item Manually mapping visual object labels with the sound event labels to be used for automated annotation
	\item Documenting (this includes the final report and the presentations)
	\item Testing the method against manual annotation
\end{itemize} 

\section{The advanced goal}

The advanced goal is meant to be worked on in case the basic goal is achieved and there is still time for further development. The advanced goal requirements are planned so that each project member gains 6 credits for the project work. This will amount to a total of 160 hours of work for each project member. 

The objectives of the advanced goal are as follows

\begin{itemize}
	\item Everything listed under the basic goal requirements
	\item Designing more complicated scenes to be recorded and testing the system with the
	\item Further developing the video detection part of the application so that different detection models could be used dynamically (as configured by the user)
	\item Testing if the use of tracking algorithms could be used to further improve the accuracy of the application (if needed)
	\item Developing a graphical user interface for the mapping of visual object labels and sound event labels
\end{itemize} 

\section{Deliverables}

The deliverables for the project will be as this project plan document, the final report of the project work, the presentations given during the course and the source codes of the developed application. 

\chapter{Project Resources}
\label{ch:project_resources}
There are multiple resources available for the project. For video object detection and tracking, there are multiple
readymade implementations available for state-of-the-art models. The same applies for many of the audio-related
tasks of the project. Furthermore, the clients are experts in audio processing and they are able to provide useful
insights throughout the project. There is also lots of video and image processing expertise at the university, so it
is possible to ask help from some expert in those fields.

Due to the ongoing pandemic, the work will be mostly conducted remotely. However, the recording process needs to be
performed in different locations determined by the group members. For these recording sessions, a 360 camera with an
integrated microphone array will be provided by the university.

\chapter{Implementation}
\label{ch:implementation}
The tasks listed in the project objectives chapter will be further divided into smaller subtasks here to give a more detailed desicription of the tasks and what are the steps to complete the planned tasks. 

\section{Project initialization and planning (basic)}

The task is to get started with the project by discussing the objectives of the project and different possible
methods that are used to achieve these objectives.

\section{Preliminary study of open-source video object detection and object tracking methods that could be applied in the project (basic)}

To automatically localize different objects in a scene, an automatic video object detection method is required.
Nowadays these methods are commonly deep learning-based methods with a plethora of different classes. The task is
to search for a method which would be most suitable for the task-at-hand by looking into the upsides and downsides
of each given model for our task.

Object tracking methods can also be utilized in the project. These methods are commonly based on mathematical 
algorithms and have been found to be quite reliable even for 360 video. However, the vast majority of these 
methods are only able to track one object at a time and this object needs to be manually selected in the initial
video frame which adds some manual work. It is possible to combine these methods together with automatic object
detectors to provide more reliable object detection by avoiding cases where a plain object detector could lose
track of an object for a few frames.

An important aspect is also to study how to automatically get the sound source activities of different objects.
This information should somehow be combined with the object detection information.

\section{Designing the scenes that are to be recorded using a 360 video camera and the integrated microphone array (basic)}

Most of the classes that are in pre-trained object detection models are irrelevant for our task. Hence, there is a
need to manually select which classes in the pre-trained model are required for the scenes that are to be 
recorded. The rest of the classes can then be combined into an `other' class label.

The scenes should be designed so that there are multiple sources of sound simultaneously. These objects should
be over two meters from the video camera. Thus, the basic assumption is that the elevation angles are limited.
Furthermore, sound sources should not be right next to each other to not make the scenes too complex.

\section{Recording the designed scenes (basic)}
After the recording location is selected, the recording of the videos starts. Since the aim is to determine 
the location and time of different sound sources in the same recording, we will cover different objects in the 
same video clips. These objects can be adults talking, a kid crying, a dog barking, a bird singing, and so on.

\section{Developing a set of python scripts for handling the video data frames (basic data handling methods for video) (basic)}

\section{Developing a set of python scripts for applying object detection for the video data frames (basic)}

\section{Developing a method for outputting the detected visual objects, their locations and the time interval of detection to be used together with spatial audio detection (basic)}

\section{Testing the developed video handling scripts with the recorded video data (basic)}

There has long been an indispensable need for object detection, recognition and tracking in computer vision
applications. YOLO is a method which is used to detect or track single or multiple objects that are in the
camera's field of view. YOLO came on the computer vision scene with the seminal 2015 paper by Joseph Redmon et al.
“You Only Look Once: Unified, Real-Time Object Detection,” and immediately got a lot of attention by fellow
computer vision researchers. In our project, YOLO will be modified such that it detects all of the necessary
objects that appear as groud truth sound sources in the recorded scenes of the project.

\section{Developing a method for outputting the detected sound events, their locations and the time interval of detection to be used together with spatial audio detection (basic)}

\section{Manually mapping visual object labels with the sound event labels to be used for automated annotation (basic)}

\section{Documenting (this includes the final report and the presentations) (basic)}

\section{Testing the method against manual annotation (basic)}

\section{More complicated scenes with more events and more sounds (advanced)}

\section{Fusion of audio and video (advanced)}

\chapter{Budget}
\label{ch:budget}
For achieving the basic goals, the estimated working time for each group member is 133.3 hours, and altogether for
all three members approximately 400 working hours. According to TEK (the largest organization for academic engineers
in Finland), the recommended salary for a starting salary for an M. Sc. (Tech.) is 3980€ in a month. There are 
approximately 172 working hours in one month, which results in 23.14€/hour. Thus, if we assume that the project
members would be starting their working careers in some company in Finland, the project would cost approximately
9256€ altogether.

For the advanced goal, the estimated working time for each group member is 160 hours, and altogether for all three
members approximately 480 working hours. Based on the aforementioned assumptions, the project would cost
approximately 11 107€ for achieving both the basic and the advanced goal.

The two clients are using approximately 30 minutes for each remote meeting which occurs every Friday. In addition,
if it is estimated that the clients both use approximately 30 minutes for the project each week outside the meeting
time, this gives an additional two working hours each week for the clients. Altogether, this leads to approximately
28 working hours for the clients. If an estimate of the salaries of the clients would be e.g. 5000€ a month, then
their input to the project would cost approximately 814€. Therefore, an estimate of the cost for the project would
be 10 070€ for only the basic goal, and 11 921€ for the basic goal and the advanced goal.

\chapter{Project Management}
\label{ch:project_management}
\section{Project management overview}

How is the project control and the management organized?


\section{Project group meeting plan}

Everything related to the project is discussed daily in a project-specific Telegram-channel.


\section{Meeting plan with the client}

A remote meeting in MS Teams is organized with the clients once a week on every Friday at 14:00. In the meeting,
updates about the project are discussed and goals for the following week are set.


\section{Risks analysis}


\section{Acceptance of results}



% Add chapters similarly.


%%%%% Bibliography/references.

% Print the bibliography according to the
% information in ./tex/references.bib and
% the in-line citations used in the body of
% the thesis.
% \emergencystretch=2em
\printbibliography[heading=bibintoc]

%%%%% Appendices.

% Use only if it clarifies the structure of
% the document. Remember to introduce each
% appendix and its content.

\begin{appendices}

\chapter{References}

\end{appendices}

\end{document}