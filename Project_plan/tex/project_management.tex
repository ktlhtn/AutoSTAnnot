\section{Project management overview}

The course schedule and the weekly objectives given by the clients impose that the start of the project is well
under control. However, after the project plan is presented at the course seminar, the management of the project
depends largely on what is agreed at the weekly discussions between the group members and the clients.


\section{Project group meeting plan}

Everything related to the project is discussed on a daily basis in a project-specific Telegram-channel.


\section{Meeting plan with the client}

A remote meeting in MS Teams is organized with the clients once a week every Friday at 14:00. In the meeting,
updates about the project are discussed and goals for the following week are set.


\section{Risks analysis}

A SWOT-analysis for the present project:
\begin{itemize}
    \item Strengths:
    \begin{itemize}
        \item All of our group members have a strong background in signal processing and machine learning.
        \item The group has previous hands-on experience in audio signal processing.
        \item The group has practical experience in data annotation.
    \end{itemize}
    \item Weaknesses:
    \begin{itemize}
        \item Spatial audio processing and 360 video processing are both completely new tasks for all group
        members.
    \end{itemize}
    \item Opportunities:
    \begin{itemize}
        \item There already exists multiple readymade implementations for many of the subtasks of the project,
        such as object detection and object tracking. These can be utilized as a basis.
        \item The clients are experts in audio processing and are able to provide useful insights.
        \item The project is a proof-of-concept, so there is no pressure from trying to obtain state-of-the-art
        performance for the entire pipeline.
    \end{itemize}
    \item Threats:
    \begin{itemize}
        \item The whole task of the project is a novel idea and thus it is possible that it not turn out to be
        very successful.
    \end{itemize}
\end{itemize}


\section{Acceptance of results}

The project will end before the end of the fourth period. The final outcome of the project will be accepted by the
clients. If there are any changes in the project plan or objectives during the project, these will be discussed
and agreed with the clients separately. The same applies for the project group members; if the structure of the
project group changes for some reason or another, then the actions that will be taken are discussed with the
clients.

If the basic goal of the project is met, then each group member will receive 5 credits each. If the advanced goal
is achieved in addition, then each group member will receive 6 credits each. These goals are determined together
with the clients.
