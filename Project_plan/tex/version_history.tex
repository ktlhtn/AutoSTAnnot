\begin{center}
	\begin{tabular}{ c c c }
		22.1.2021 & Kalle L & Initial \\ 
		24.1.2021 & Kalle L & Added version history page \\
		26.1.2021 & Mehmet A & Added abstract first draft \\ 
		27.1.2021 & Kalle L & Additions to the abstract and the template \\   
	\end{tabular}
\end{center}